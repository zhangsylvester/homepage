% --- LaTeX CV Template - S. Venkatraman ---

% Set document class and font size
\documentclass[letterpaper, 11pt]{article}
\usepackage[utf8]{inputenc}
\usepackage{CJKutf8}
\newcommand{\mycolor}[0]{\color{RoyalBlue}}

% Package imports
\usepackage{setspace, longtable, graphicx, hyphenat, hyperref, fancyhdr, ifthen, everypage, enumitem, amsmath, setspace, comment}

% --- Page layout settings ---

% Set page margins
\usepackage[left=0.8in, right=0.8in, bottom=0.7in, top=0.7in]{geometry}

% Set line spacing
\renewcommand{\baselinestretch}{1.15}

% --- Page formatting --- \color{OliveGreen}

% Set link colors
\usepackage[dvipsnames]{xcolor}
\hypersetup{colorlinks=true, linkcolor=OliveGreen, urlcolor=OliveGreen}

% Set font to Libertine, including math support
\usepackage{libertine}
\usepackage[libertine]{newtxmath}

% Remove page numbering
\pagenumbering{gobble}

% --- Document starts here ---

\begin{document}

% Name and date of last update to this document
\noindent{{\Huge{Sylvester W. Zhang}}
\hfill{\it\footnotesize Updated \today}}\vspace{0.2cm}\\
\hspace{1.5cm}\begin{CJK}{UTF8}{gkai}{\Huge张文泽 }\end{CJK}


% --- Contact information and other items ---

\vspace{0.5cm} 
% --- Start the two-column table storing the main content ---

% Set spacing between columns
\setlength{\tabcolsep}{8pt}

% Set the width of each column
\begin{longtable}{p{1.5in}p{4.9in}}
% --- Personal info

\nohyphens{\mycolor{Personal Information}}& \textbf{Email}: \href{mailto:swzhang@umn.edu}{swzhang@umn.edu}\\
&\textbf{Homepage}: \href{http://www.sylvesterzhang.com}{http://www.sylvesterzhang.com}\\
&\\


% --- Section: Research interests ---

\nohyphens{\mycolor{Research Interests}}
& Algebraic combinatorics. \\
& \\

% --- Section: Education ---

\mycolor{Education} 
& \textbf{University of Minnesota, Twin Cities} \hfill Minneapolis, MN \\ 
&\ \ \ Ph.D. in Mathematics \hfill Sep 2020 -- Present \\
&\ \ \ Advisor: Pavlo Pylyavskyy\\ 
%\vspace{cm}
%& \\
%& \textbf{University of Minnesota, Twin Cities} \hfill City, State \\
& \textbf{University of Minnesota, Twin Cities} \hfill Minneapolis, MN \\ 
&\ \ \ B.S. in Mathematics \hfill Sep 2016 -- May 2020\\
%& Mentors: Professors C, D. {\it GPA: X.YZ.}\\
%& \\
%& \textbf{University 3} \hfill City, State\\
&\ \ \ B.A. in Quantitative Economics %\hfill Sep 2016 -- May 2020 
\\
%& Mentors: Professors E, F. {\it GPA: X.YZ.}\\
%& \\
% --- Uncomment the next few lines if you want to include some courses ---
%& \textbf{Selected coursework}
%\begin{itemize}[noitemsep,leftmargin=*]
%\item \underline{Relevant subject 1}: Course 1, Course 2, Course 3, Course 4
%\item \underline{Relevant subject 2}: Course 1, Course 2, Course 3, Course 4
%\end{itemize} \\

% --- Section: Awards, scholarships, etc. ---
% --- Note: section title is spread over two lines ---
\begin{comment}
{\mycolor{Awards and}} 
& UROP Scholarship (University of Minnesota) \hfill 2020\\
{\mycolor{scholarships}} 
& Name of award 2 (Organization that gave you the award)\hfill 2019 \\
& Name of award 3 (Organization that gave you the award) \hfill 2018 \\
\end{comment}
& \\


% --- Section: Publications ---
%\nohyphens{\mycolor{Preprints}}
%&\begin{enumerate}
%\item {J. Wellman, Q. V. Dao, S. Zhang, C. Yost-Wolff. Rowmotion Orbits of Trapezoid Posets. arXiv preprint \href{https://arxiv.org/abs/2002.04810}{arXiv:2002.04810}, 2020.}
%\item {S. Chepuri, CJ Dowd, A. Hardt, G. Michel, \textbf{S. Zhang}, V. Zhang. Arborescence of Covering Graphs. arXiv preprint \href{https://arxiv.org/abs/1912.01060}{arXiv:1912.01060}, 2019.}
%\end{enumerate}

\nohyphens{\mycolor{Preprints}} 
&\textbf{An Expansion Formula for Decorated Super-Teichm\"uler Spaces.} \\
&\ \ \ with  G. Musiker \& N. Ovenhouse. \\
&\ \ \ arXiv preprint \href{https://arxiv.org/abs/2102.09143}{arXiv:2102.09143}, 2021.\\

& {\textbf{Rowmotion Orbits of Trapezoid Posets.} } \\
&\ \ \ with J. Wellman, Q. Dao, \& C. Yost-Wolff.\\
&\ \ \ arXiv preprint \href{https://arxiv.org/abs/2002.04810}{arXiv:2002.04810}, 2020.\\

&\textbf{Arborescence of Covering Graphs.} \\
&\ \ \ with S. Chepuri, C. Dowd, A. Hardt, G. Michel, \& V. Zhang.\\
&\ \ \  arXiv preprint \href{https://arxiv.org/abs/1912.01060}{arXiv:1912.01060}, 2019.\\
%& \textbf{Title of your third most recent research paper} \\
& \\
\nohyphens{\mycolor{Work in Progress}} 
&\textbf{A Lattice Model for LLT Polynomials.}\\
&\ \ \ with M. Curran, C. Yost-Wolff \& V. Zhang. \\
& \\
% --- Section: Talks and tutorials ---

{\mycolor{Talks}} 
& \textbf{Schur and LLT Polynomials from Lattice Models.} \hfill March 2021 \\
&\ \ \ Graduate Online Combinatorics Colloquium (GOCC) \\
& \textbf{$T$-paths Formula for Decorated Super-Teichm\"uller Spaces.} \hfill Feb 2021 \\
&\ \ \ Combinatorics Seminar, University of Minnesota \\

%& \\
& \\



% --- Section: Teaching experience ---

{\mycolor{Teaching Experience}} 
& \textbf{Teaching assistant, University of Minnesota}  \\
&\ \ \ Math 1271: Fall 2020 Spring 2021\\
&\ \ \ Math 1051: Fall 2019 Spring 2020 \\

& \\


% --- Section: Various skills (programming, software, languages, etc.) ---

{\mycolor{Skills}} 
& \textbf{Programming}\\
&\ \ \ Python, SageMath, Mathematica. \\
& \textbf{Languages} \\
&\ \ \ Chinese Mandarin (native), English (fluent) \\
%& \\

% --- Section: Service and outreach ---

%\mycolor{Service and outreach}
%& \textbf{Title of organization you were in} \hfill Month Year -- Month Year \\
%& Description of your responsibilities. Integer pretium semper justo. Proin risus. Aliquam volutpat est vel massa. \\
%& \\

% --- Section: Professional society memberships ---
% --- Note: section title is spread over two lines ---

%{\mycolor{Professional}} 
%& {\textbf{Name of professional society.}} \hfill Month Year -- Present \\
%{\mycolor{memberships}} 
%& Some things you did or conferences you attended. Aliquam volutpat est vel massa. Sed dolor lacus, imperdiet non, ornare non, commodo eu, neque. \\
%& \\

% --- Section: Other interests/hobbies ---

%\nohyphens{\mycolor{Other interests}} & Some of your hobbies etc.\\

% --- End of CV! ---
\end{longtable}
\end{document}